\section{Analiza sistema}
Cilj ovog sistema jeste da olakša funkcionisanje jednog on-line restorana. To se postiže međusobnom komunikacijom korisnika, koordinatora kao i osoba zaduženih za pripremu i distribuciju hrane. Na taj način realizuju se zahtevi korisnika u najkraćem vremenskom roku sa što boljim kvalitetom usluga.

%Naime, korisniku je omogućeno da hranu naruči bilo putem telefona, bilo putem veb stranice.


%Ukoliko korisnik prvi put pristupa veb stranici radi naručivanja, neophodno je da se registruje sa ličnim podacima (ime, prezime, adresa, korisničko ime, lozinka...) dok svaki sledeći pristup stranici omogućen je običnom prijavom. Pored uobičajenih porudžbina, postoji i mogućnost aranžiranja i isporuke keteringa za različite tipove proslava.


%Sam restoran poseduje i sopstveni magacin za skladištenje svih potrebnih namirnica. Uz to, obezbeđeno je svo neophodno osoblje za koordinisanje ovakvog tipa restorana.


\subsection{Učesnici}
Glavna podela učesnika sistema je na:

\begin{itemize}
    \item Korisnik 
    \item Koordinator
    \item Kuvar
    \item Dostavljač
    \item Magacioner
    \item Menadžer
    \item Dekorater
    \item Dobavljač
\end{itemize}
