\newpage
\section{Opis baze podataka}
Informacioni sistem  restorana osmišljen je tako da umnogome olakša poslovanje samog restorana, rad zaposlenih u njemu kao i komunikaciju izmedju zaposlenih. Jedan od najbitnijih aspekata ovog informacionog sistema je upravo baza podataka. Pravilno dizajnirana baza podataka pruža njegovim korisnicima pristup ažuriranim, tačnim informacijama. \\
\indent Baza podataka je projektovana tako da pokriva sve slučajeve upotrebe informacionog sistema. \\
\indent U bazi postoji apstraktni tip eniteta \textbf{osoba} iz kog su izvedeni tipovi entiteta  \textbf{zaposleni} i \textbf{korisnik} koji opisuju redom skup zaposlenih u restoranu i skup registrovanih korisnika restorana, tim redom. \\

Tabela \textbf{osoba} sadrži zajedničke informacije za zaposlene i korisnike. Polje "id\_osobe" se odnosi na jedinstveno korisničko ime, a "aktivan" označava da li je zaposleni korisnik trenutno aktivan (bit 1) na nalogu restorana ili ne (bit 0). \\

\indent Registracijom korisnika na sajtu restorana dodaje se red u tabeli \textbf{oso\-ba}. Apstraktni entitet \textbf{osoba} se specijalizuje entitetom \textbf{korisnik} i tabela \textbf{korisnik} se popunjava odgovarajućim redom. \\

Slično, kreiranjem naloga za novog radnika se apstraktni entitet \textbf{osoba} specijalizuje entitetom \textbf{zaposleni} i tabela \textbf{zaposleni} se popunjava odgovarajućim redom. Ova tabela povezana je sa tabelom \textbf{uloga} ključem koji ukazuje na odgovarajuću poziciju u sistemu zaposlenih.\\

Deaktivacija naloga otpuštenog radnika vrši se samo promenom bita sa 1 na 0 u polju "aktivan" tabele \textbf{osoba}.\\

Svaki korisnik ima pravo da ostavi utisak o kvalitetu usluge restorana. Ocenjvanje usluge restorana vrši se na osnovu polja "Jeloid\_jela" u tabeli \textbf{ocena}. To je ujedno i strani kluč koji se odnosi na tabelu \textbf{jelo}. Polje "vrednost" ukazuje na ocenu, a komentar nije obavezan. Tabela \textbf{ocena} je povezana stranim ključem "KorisnikOsobaid\_osobe" i sa tabelom \textbf{korisnik}. \\

Prilikom naručivanja (onlajn ili telefonom) dodaje se novi red u tabeli \textbf{porudžbina} pod uslovom da je porudžbina prihvaćena. Tabela je povezana sa tabelom \textbf{korisnik} identifikatorom osobe kao i sa tabelom \textbf{spisak} identifikatorom porudžbine. Da bi se proverila mogućnost realizacije porudžbine, dodaje se najpre novi red u tabeli \textbf{spisak} sa traženim zahtevima. Koordinator proverava preko identifikatora jela i namirnica da li ima dovoljno zaliha trenutno u magacinu. Identifikatori jela i namirnica su povezani tabelom \textbf{priprema}. Napomena: identifikator jela se odnosi kako na hranu, tako i na piće.